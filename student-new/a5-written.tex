% Created 2021-05-26 Wed 09:19
% Intended LaTeX compiler: pdflatex
\documentclass[11pt]{article}
\usepackage[utf8]{inputenc}
\usepackage[T1]{fontenc}
\usepackage{graphicx}
\usepackage{grffile}
\usepackage{longtable}
\usepackage{wrapfig}
\usepackage{rotating}
\usepackage[normalem]{ulem}
\usepackage{amsmath}
\usepackage{textcomp}
\usepackage{amssymb}
\usepackage{capt-of}
\usepackage{hyperref}
\author{Alok Regmi\thanks{sagar.r.alok@gmail.com}}
\date{\today}
\title{A5 Written}
\hypersetup{
 pdfauthor={Alok Regmi},
 pdftitle={A5 Written},
 pdfkeywords={},
 pdfsubject={},
 pdfcreator={Emacs 28.0.50 (Org mode 9.5)}, 
 pdflang={English}}
\begin{document}

\maketitle
\tableofcontents



\section*{Q.1.a.}
\label{sec:orgf864f36}

Copying in attention. Describe (in one sentence) what properties of the inputs to the attention operation would result in
the output c being approximately equal to vj for some j ∈ \{1, . . . , n\}. Specifically, what must be
true about the query q, the values \{v1 , . . . , vn \} and/or the keys \{k1 , . . . , kn \}?

\subsection*{\acr{answer}}
\label{sec:org8311e15}

Since our softmax function never gives output that's exactly 0 to all the elements, we will copy our v\textsubscript{j} into the attention output only if our value vector is represented as one-hot vector.
\section*{Q.1.b.}
\label{sec:org4c8d08e}

Assume key vectors as perpendicular vectors and values be arbitrary. Let two values from value vectors be \(v_{a}\) and \(v_{b}\) . Give expression for query vector q such that the output c is approximately equal to average of the two.

\subsection*{\acr{answer}}
\label{sec:org3e76666}

\begin{itemize}
\item This has to be related to our keys. Keys are independent of each other.
\item We need not scale \([k_{a}, k_{b}]\) since it's already assumed that \(||k_{I}||\) = 1.
$$ q = \frac{k_{a} + k_{b}}{2}$$

 Then,
 $$ qk^{T} = [k_{a}.q, k_{b}.q, .... , k_{i}.q]$$
Since q is linear combination of two vectors \(k_{a}\) and \(k_{b}\), all the dot products except for \(k_{a}\) and \(k_{b}\) will be 0.
 Thus, $$qk^{T} = [\frac{_{}k_{a}.k_{a}}{2}, \frac{k_{b}.k_{b}}{2}, 0,0,....,0]$$
 Now alpha will be almost non-negligible for all the values that are 0. We can scale up the vector by scalar \(s\) if required so that the probabilities get close to 0.5.
\end{itemize}
\section*{Q.1.c.i}
\label{sec:org5c0b914}

Now assuming key vectors are randomly sampled \(k_{i} \sim \mathcal{N}(\mu_{i}, \sum_{i})\) with means \(\mu_{i}\) known but covariances \(\sum_{i}\) unknown. Further, all means \(\mu_{i}\) are perpendicular and unit norm. \(||\mu_{i}|| = 1\).

Further assume, covariance matrices \(\sum_{i} = \alpha I\), for vanishingly small \(\alpha\).

\section*{Q.1.c.ii}
\label{sec:orgf06edcc}
\end{document}
